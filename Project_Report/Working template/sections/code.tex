\chapter{Explaining the Codes}
Here we'd be briefly going through the flow of the codes written for \arduino{} and in \emph{Processing}. The entire Arduino and Processing codes are given (explained with comments) in Appendix \ref{ch:appAlabel} and Appendix \ref{ch:appBlabel}, respectively.
\section{The Arduino Code}
Every \arduino{} code consists of two standard functions - \textit{void setup()} and \textit{void loop()}. The former is used to define and initialise variables, pins and the baud rate of serial communication.\\
We have programmed the Arduino to control the circuit in the following flow:
\begin{enumerate}
	\item Rotate the motor to an angle,
	\item Use the ultrasonic sensor and calculate the distance,
	\item Send both the above values via serial communication,
	\item Rotate the motor to the next degree,
	\item Repeat steps 2-5.
\end{enumerate}
The following code snippets implement the above given steps:
\begin{mdframed}[backgroundcolor=light-gray, roundcorner=10pt,leftmargin=1, rightmargin=1, innerleftmargin=15, innertopmargin=15,innerbottommargin=15, outerlinewidth=1, linecolor=light-gray]
	\begin{lstlisting}[caption={Step 1},language = C]
	// rotates the servo motor from init_ang to fin_ang degrees
	for(int i=init_ang;i<=fin_ang;i++)
	{  
	myServo.write(i-offset);  //angle value to be passed to the servo library object for writing into the motor
	delay(17);//DELAY #1:for time taken in motor rotation for one degree before calculating distance
	\end{lstlisting}
	
\begin{lstlisting}[caption={Step 2},language = C]
	distance = calculateDistance();// Calls a function for calculating the distance measured by the Ultrasonic sensor for each degree
	\end{lstlisting}
	
\begin{lstlisting}[caption={Function called above},language = C]
	// Function for calculating the distance measured by the Ultrasonic sensor
	float calculateDistance(){ 
	unsigned long T1 = micros();
	digitalWrite(trigPin, LOW); // trigPin needs a fresh LOW pulse before sending a HIGH pulse that can be detected from echoPin
	delayMicroseconds(2);//DELAY #2:time for which low trig pulse is maintained before making it high
	digitalWrite(trigPin, HIGH); 
	delayMicroseconds(10);//DELAY #3:Sets the trigPin on HIGH state for 10 micro seconds
	digitalWrite(trigPin, LOW);
	duration = pulseIn(echoPin, HIGH); // Reads the echoPin, returns the sound wave travel time in microseconds
	//distance= duration*0.034/2;
	distance = (duration/2)/29.1;     //in cm,  datasheet gives "duration/58" as the formula
	\end{lstlisting}
	
\begin{lstlisting}[caption={Step 3},language = C]
	Serial.print(i); // Sends the current degree into the Serial Port for graphical representation
	Serial.print(","); // Sends addition character right next to the previous value needed later in the Processing IDE for indexing
	Serial.print(distance); // Sends the distance value into the Serial Port for the graph
	Serial.print(";"); // Sends addition character right next to the previous value needed later in the Processing IDE for indexing
\end{lstlisting}
\end{mdframed}
\vspace{1cm}
Keeping the above code in a \texttt{void loop(){}} ensures that the steps get repeated in a loop, as desired.
\clearpage
\section{The Processing Code}
The JAVA code written for processing divides its work into a few main functions, which are customary for every code written in Processing. The functions are
\begin{itemize}
	\item \begin{verbatim}void setup()\end{verbatim} - to initialise variables and baud rate of serial communication,
	\item \begin{verbatim}void draw()\end{verbatim} - to make a 2D/3D drawing on the screen on which to show the output,
	\item \begin{verbatim}void serialEvent()\end{verbatim} - the function dealing with incoming/outgoing serial data, and operations on it.
\end{itemize}
Customised functions were used to draw several parts of the displayed graph, which were called from \texttt{void draw()}. These are :
\begin{itemize}
	\item \begin{verbatim}void drawRadar()\end{verbatim} - to draw the basic circular layer of graphics
	\item \begin{verbatim}void drawLine()\end{verbatim} - to draw a line with trails which would be used to sweep over the area implying scanning
	\item \begin{verbatim}void drawObject()\end{verbatim} - to draw a colored line over the scanning figure to indicate presence of an object, whose length would be dependant on the distance of the object from the sensor
	\item \begin{verbatim}void drawText()\end{verbatim} - to place text at different places on the map to show the serial and calculated data values
\end{itemize}
\clearpage